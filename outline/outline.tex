\documentclass[11pt,preprint]{aastex}
\pagestyle{myheadings}

\input{../../../papers/00_bibdatabase/astromacro}
% Macro for Paragraph Outlines
\input{../../../papers/00_bibdatabase/outlinemacro}

%% comment out either draft true or false
\newboolean{draft}
\setboolean{draft}{true}
%\setboolean{draft}{false}

% comment out \draftmodep line to remove all commnts and notes from
% manuscript
%\draftmodeptrue

\begin{document}

%\NOTE{Outline}
%\NS{test}

Memo:
\begin{itemize}
\item Ver 1.0 : AG First Draft (04/04/2008 4:20 pm at LBL)
\end{itemize}

% This is for our note
\AG{Please check}

% Citation Test
\citet{perlmutter98a}
\citet{suzuki05a}

\outlinestart{draft}

\section{Introduction: 0.8 page}
\outline{draft}{The Large Number of Quasar Spectra, and its Variety}
\outline{draft}{Objective Classification, PCA}
\outline{draft}{Discovery of Weak Emission Lines in the Forst}
\outline{draft}{Guide Lines for Continuum Fitting in the Ly alpha forest}
\outline{draft}{Artifitial Spectra}

\section{Data}
\outline{draft}{Bechtold 334 HST FOS spectra, continua fitted, BAL removed}
\outline{draft}{S/N, binning, redshift correction, normalization}
\outline{draft}{Prediction Summary}

\section{PCA \& PCA components}
\outline{draft}{A brief history of PCA}
\outline{draft}{PCA formulation}
\outline{draft}{Principal Component Spectra, figure}
\outline{draft}{Variance, table}
\outline{draft}{PCS Coefficients}

\section{PCA classification}
\subsection{Weak Emission in the Ly alpha Forest}
\subsection{Class Zero, Mean Spectra, figure}
\subsection{Class I}
\subsection{Class II}
\subsection{Class III}
\subsection{Class IV}

\section{Luminosity Relation}
The essence of the observational astrophysics is to find the correlation
between the observables and the physical properties.
\outline{draft}{Baldwin Effect}
\outline{draft}{Hint of Correlation}
\outline{draft}{Finding Luminosity, equations}
\outline{draft}{3 steps \& Projection Matrix, table}
\outline{draft}{Testing Prediction}
\outline{draft}{Comments on Prediction}

\section{Artifitial Spectra}
\outline{draft}{Active use of PCS}
\outline{draft}{How to generate Artifitial Spectra}
\outline{draft}{Use of Artifitial Spectra}

\section{Future Prospective}
\outline{draft}{How to improve, wavelength range, number of spectra}
\outline{draft}{What can be done, supernova}


\outlineend{draft}

\bibliographystyle{../../../papers/00_bibdatabase/apj}
\bibliography{../../../papers/00_bibdatabase/archive}

\end{document}
