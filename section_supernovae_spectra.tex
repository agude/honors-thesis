\section{Spectra of Type Ia Supernovae}
% More intro
Optical spectra are the means through which supernovae are classified \citep{branch01a}. Type Ia spectra are characterized by a deep absorption feature near 6150 \AA\ produced by blue shifted Si II $\lambda$6347, $\lambda$6371. The early spectrum exhibits broad features from lines of neutral and singly ionized intermediate-mass elements including O, Mg, Si, S, and Ca. There is some contribution from iron-peak elements, primarily Fe and Co near UV wavelengths. At this point the strongest features are those that arise from SI II $\lambda$6355 and Ca II H\&K $\lambda$3934 and $\lambda$3968. As the spectrum evolves in time Fe II lines becomes prominent, and the evolution slows down.

The spectra of type Ia supernovae are rather homogeneous if compared at the same phase and can be used to estimate time of max if compared with spectral templates \citep{filippenko97a}. There are however differences in the spectra of type Ia supernovae. Line depths vary between different supernovae as does the velocity of ejecta. %%% A spectroscopic survey of type Ia's has concluded that populations in elliptical galaxies differ from those hosted in spiral galaxies, and that this difference can only be accounted for by actually differences in the supernovae and not differences in viewing conditions. %(Note: Fileppenko 1997, but unclear which source he refers to).

One class of peculiar supernovae are the SN1991T-like supernovae. These supernovae are overly luminescent and bluer in B-V color than normal type Ia supernovae. They prominently feature a high excitation Fe III feature near maximum, with the characteristic type Ia features developing after maximum. These species of type Ia do not exhibit Si II or Ca II absorption lines in their early spectra.

A second peculiar class of supernovae are the SN1991bg-like supernovae. These supernovae are characteristically sub-luminous in V and B, and are particularly red in B-V color. After maximum they decline more quickly then the typical type Ia \citep{filippenko92b}. These supernovae have a deep absorption feature at $\lambda$4200 from Ti II.
