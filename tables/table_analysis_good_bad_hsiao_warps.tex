%% The values (usually only l,r and c) in the last part of
%% \begin{deluxetable}{} command tell LaTeX how many columns
%% there are and how to align them.
\begin{deluxetable}{cc}

%% Keep a portrait orientation

%% Over-ride the default font size
%% Use Default (12pt)

%% Use \tablewidth{?pt} to over-ride the default table width.
%% If you are unhappy with the default look at the end of the
%% *.log file to see what the default was set at before adjusting
%% this value.

%% This is the title of the table.
\tablecaption{Result Of Warping Hsiao Template With Cardelli Law}

%% This command over-rides LaTeX's natural table count
%% and replaces it with this number.  LaTeX will increment
%% all other tables after this table based on this number
\tablenum{4}

%% The \tablehead gives provides the column headers.  It
%% is currently set up so that the column labels are on the
%% top line and the units surrounded by ()s are in the
%% bottom line.  You may add more header information by writing
%% another line between these lines. For each column that requries
%% extra information be sure to include a \colhead{text} command
%% and remember to end any extra lines with \\ and include the
%% correct number of &s.
\tablehead{\colhead{Well fit supernovae} & \colhead{Badly fit supernovae} \\
\colhead{} & \colhead{} } % for units

%% All data must appear between the \startdata and \enddata commands
\startdata
SN1997dt & SN1991T \\
SN1998aq & SN1991bg \\
SN1999dh & SN1998bp \\
SN1998V  & SN1998de \\
SN1998eg & 
SN1998ec & SN1999ac \\
SN1999aa & SN1999cc \\
SN1999dq & SN1999cl \\
SN1999gd & SN1999ej \\
SN1999gp & SN2000cx \\
SN2000cf & SN2000dk \\
SN2000fa \\
SN2000es
\enddata

%% Include any \tablenotetext{key}{text}, \tablerefs{ref list},
%% or \tablecomments{text} between the \enddata and
%% \end{deluxetable} commands

%% No \tablecomments indicated

%% No \tablerefs indicated

\end{deluxetable}

