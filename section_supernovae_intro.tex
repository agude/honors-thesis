\section{Supernovae Cosmology}
\subsection{A History of Type Ia Supernovae as Standard Candles}
Type I supernovae have been used as distance indicators for over forty years. They were first used by \cite{kowal68a} when he published a type I supernova Hubble diagram. Originally a type I supernova was defined as any supernova with a lack of hydrogen in their optical spectra \citep{minkowski41a}. It has since become clear that type I supernovae can actually be subdivided into two distinct classes: type Ib/c which are generated by massive stats that undergo a core collapse and type Ia which it is theorized are thermonuclear explosions of white dwarfs. 

By the late 1980s it was becoming clear that most type Ia supernovae had similar spectral time series, light curves, and absolute magnitudes at maximum light. In 1992 a review by \citeauthor*{branch92a} concluded that type Ia SNe were "the best standard candles known so far", with a dispersion in maximum B and V band magnitudes that was $< 0.25$ mag and likely even smaller. 

The quality of data continued to improve over the next few years. The Calan/Tololo Supernova Search (CTSS) in 1990 obtained a set of high quality light curves and spectra of supernovae in the range $z = 0.01 - 0.1$ which allowed them to compare peak magnitudes while calculating their relative distance through their Hubble velocities \citep{hamuy93a}. The search was difficult because as the appearance of a supernova is unpredictable, the team was unable to schedule follow up observations until after a supernova was found. Despite the challenges, CTSS was able to acquire thirty new type Ia light curves \citep{hamuy05a}.

With the wealth of new data, several methods were devised to select the most easily calibrated type Ia supernovae from a set. \citet{vaughan95a} developed $B-V$ color cuts that selected type Ia supernovae that had an observed dispersion of less than $0.25$ mag. \citet{phillips93a} discovered a relation between absolute magnitude and $\Delta m_{15} (B)$, the amount the supernova decreased in brightness in the B-band over a 15 day period following maximum light.

Based on the success of the $\Delta m_{15} (B)$ parameter, \citet{Riess96b} developed the multi-color light curve shape method (MLCS). This method parametrized light curves as a function of their absolute magnitude at maximum and fit for all colors simultaneously. By fitting all colors at once the MLCS method allowed color excess, $E(B-V)$ to be calculated. Traditionally this excess has been attributed to intervening dust which reddens the supernova, and so has been used to correct for extinction \citep{riess96c}. % CHECK THIS REF!!!

\citeauthor{perlmutter99a} (\citeyear{perlmutter97a} and \citeyear{perlmutter99a}) developed their own method of parameterizing the B and V bands of a light curve. Using a stretch factor, a measure of the amount a canonical light curve needs to be stretched in time to match the observed light curve, they were able to more simply represent a light curve.

Type Ia supernovae can be used as standard candles because although they are not of a perfectly uniform luminosity, the above methods can be used to to calibrate them \citep{perlmutter03a}. Further there are indications that there exists tight correlations between the spectral features of specific supernovae and their peak luminosity that should allow even more accurate calibration. It is already known that the ratio of the Si II feature at $\lambda$5750 to the feature at $\lambda$6150 increases with decreasing luminosity \citep{nugent95a}. Likewise, the ratio of the two peaks on either side of Ca II H\&K absorption share a similar relationship \citep{filippenko97a}.
